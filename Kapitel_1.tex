
\chapter{Einführung}
\label{sec:intro}
	
	Im Laufe der Zeit werden immer mehr Prozesse digitalisiert bzw. 
	automatisiert - sei es in der Industrie oder im zwischenmenschlichen 
	Bereich oder auch im Privathaushalt. Immer mehr Aufgabengebiete sollen vom 
	Menschen auf die Maschine übergehen, so auch der Vorgang des Sehens oder 
	auch des Erkennens von bestimmten Mustern. Gerade auf der Seite der 
	Industriebetriebe besteht ein Interesse an einer maschinellen 
	Fertigstellung bestimmter Arbeitsschritte. Was dies aus 
	volkswirtschaftlicher Sicht bedeutet, soll an dieser Stelle jedoch nicht 
	diskutiert werden. \\
	Viel mehr soll hier auf einen spezifischen Aspekt des maschinellen Sehens 
	eingegangen werden - die Bildverarbeitung im Allgemeinen und die 
	Segmentierung im Speziellen.
	
	\section{Ziel}
	\label{sec:ziel}
	
		Nun, nachdem der theoretische Grundstein gelegt wurde, lässt sich die Intention dieser Arbeit formulieren:
		\begin{itemize}
			\item Sie soll erstens die Problemstellung in einer klaren Form präsentieren (Abschnitt \ref{sec:motivation}) sowie in ein möglichst passendes mathematisches Modell unter Einbettung von \gls{de} übersetzen (Abschnitt \ref{sec:prob}).
			\item Zweitens wird eine Umsetzungsstrategie zur Implementierung dieses Modells beschrieben (Abschnitt \ref{sec:prob}).
			\item Schlussendlich sollen die Testergebnisse dieser Umsetzung in Kapitel \ref{sec:ex} veranschaulicht werden. Darauf aufbauend findet eine Bewertung hinsichtlich der Tauglichkeit der vorgestellten Strategie zur Lösung der eingangs in Abschnitt \ref{sec:motivation} geschilderten Problemstellung statt (Kapitel \ref{sec:sumary}). Zudem wird ein Ausblick auf mögliche Verbesserungsansätze geboten. 
		\end{itemize}
	
	\section{Motivation}
	\label{sec:motivation}