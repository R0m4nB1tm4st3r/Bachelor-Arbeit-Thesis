
\chapter{Einführung}
\label{sec:intro}
	Dieses Kapitel soll einen Überblick über Motivation, Ziel sowie Struktur der vorliegenden Arbeit verschaffen. Außerdem wird ein theoretisches Basiswissen in den Bereichen Bildverarbeitung und Evolutionäre Algorithmen bereitgestellt, welches notwendig ist, um den Kontext dieses Erzeugnisses greifbar zu machen. 
	
	
	
	\section{Ziel}
	\label{sec:ziel}
	
		Nun, nachdem der theoretische Grundstein gelegt wurde, lässt sich die Intention dieser Arbeit formulieren:
		\begin{itemize}
			\item Sie soll erstens die Problemstellung in einer klaren Form präsentieren (Abschnitt \ref{sec:motivation}) sowie in ein möglichst passendes mathematisches Modell unter Einbettung von \gls{de} übersetzen (Abschnitt \ref{sec:prob}).
			\item Zweitens wird eine Umsetzungsstrategie zur Implementierung dieses Modells beschrieben (Abschnitt \ref{sec:prob}).
			\item Schlussendlich sollen die Testergebnisse dieser Umsetzung in Kapitel \ref{sec:ex} veranschaulicht werden. Darauf aufbauend findet eine Bewertung hinsichtlich der Tauglichkeit der vorgestellten Strategie zur Lösung der eingangs in Abschnitt \ref{sec:motivation} geschilderten Problemstellung statt (Kapitel \ref{sec:sumary}). Zudem wird ein Ausblick auf mögliche Verbesserungsansätze geboten. 
		\end{itemize}
	
	\section{Motivation}
	\label{sec:motivation}