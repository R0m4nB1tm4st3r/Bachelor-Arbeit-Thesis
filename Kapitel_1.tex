
\chapter{Einführung}
\label{sec:intro}
	
	Im Laufe der Zeit werden immer mehr Prozesse digitalisiert bzw. 
	automatisiert - sei es in der Industrie, im zwischenmenschlichen 
	Bereich oder auch im Privathaushalt. Immer mehr Aufgabengebiete sollen vom 
	Menschen auf die Maschine übergehen, so auch der Vorgang des Sehens oder 
	auch des Erkennens von bestimmten Mustern. Dies ist für Industriebetriebe 
	insofern interessant, da es beispielsweise eine automatisierte 
	Identifizierung von Bauteilen anhand eines darauf eingravierten 
	Zeichencodes ermöglicht. Abbildung 'hier Bildreferenz einfügen' illustriert 
	exemplarisch ein derartiges Bauteil:\\
	
	hier Bild einfügen \\
	
	
	
	Dabei ist innerhalb dieser Arbeit die folgende Struktur vorgesehen: \\
	Kapitel \ref{sec:prob} behandelt hierbei die 
		
	\section{Ziel}
	\label{sec:ziel}
	
		Nun, nachdem der theoretische Grundstein gelegt wurde, lässt sich die Intention dieser Arbeit formulieren:
		\begin{itemize}
			\item Sie soll erstens die Problemstellung in einer klaren Form präsentieren (Abschnitt \ref{sec:motivation}) sowie in ein möglichst passendes mathematisches Modell unter Einbettung von \gls{de} übersetzen (Abschnitt \ref{sec:prob}).
			\item Zweitens wird eine Umsetzungsstrategie zur Implementierung dieses Modells beschrieben (Abschnitt \ref{sec:prob}).
			\item Schlussendlich sollen die Testergebnisse dieser Umsetzung in Kapitel \ref{sec:ex} veranschaulicht werden. Darauf aufbauend findet eine Bewertung hinsichtlich der Tauglichkeit der vorgestellten Strategie zur Lösung der eingangs in Abschnitt \ref{sec:motivation} geschilderten Problemstellung statt (Kapitel \ref{sec:sumary}). Zudem wird ein Ausblick auf mögliche Verbesserungsansätze geboten. 
		\end{itemize}
	
	\section{Motivation}
	\label{sec:motivation}