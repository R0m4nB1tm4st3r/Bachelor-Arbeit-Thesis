

\vspace*{4cm}
	\section*{\centerline{Abstract}}
	\pagenumbering{arabic}
	\par\smallskip
	\vspace*{1cm}
	Die Bildverarbeitung übernimmt im zunehmenden Maße Aufgaben, welche gerade bei großen Stückzahlen für den Menschen mit immensem zeitlichen Aufwand verbunden sind. Jene Aufgaben beinhalten in der Regel das Erkennen be\-stimm\-ter Begebenheiten oder Objekte auf Bauteilen in der Industrie. Al\-ler\-dings stellt es bisweilen eine große Herausforderung dar, einer Maschine beizubringen, was ein Mensch permanent bewerkstelligt - das Erkennen von Mustern.\\
	Daher wird beständig nach Möglichkeiten gesucht, die einzelnen Bearbeitungsschritte in der Bildverarbeitung mit Hinblick auf Zuverlässigkeit und Performance zu verbessern. 
	\par\smallskip 
	So soll im Rahmen dieser vorliegenden Arbeit eine Methode aus den Reihen der evolutionären Algorithmen, welche in Kapitel \ref{sec:intro} näher erläutert werden, im Bezug auf die \textit{Segmentierung}\footnote{\textit{Segmentierung}: Ein Arbeitsschritt in der Bildverarbeitung, bei dem die zu lesende Information im Bild vom Hintergrund farblich getrennt wird.} erprobt werden. Als Versuchsobjekte dienen metallische Platten, worauf ein Zeichencode graviert ist, der erkannt werden soll.