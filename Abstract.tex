
\glsreset{ea}
\vspace*{4cm}
	\section*{\centerline{Abstract}}
	\pagenumbering{Roman}
	\par\smallskip
	\vspace*{1cm}
	Die industrielle Bildverarbeitung übernimmt im zunehmenden Maße Aufgaben 
	bei der Fertigung von Produkten, welche gerade bei großen Stückzahlen für 
	den Menschen mit immensem zeitlichen Aufwand verbunden sind. Dies betrifft 
	in der Regel das Erkennen be\-stimm\-ter Begebenheiten oder Objekte auf 
	Bauteilen. Al\-ler\-dings stellt es bisweilen eine große Herausforderung 
	dar, einer Maschine beizubringen, was ein Mensch permanent bewerkstelligt - 
	das Erkennen von Mustern.\\
	Daher wird beständig nach Möglichkeiten gesucht, die einzelnen Bearbeitungsschritte in der Bildverarbeitung mit Hinblick auf Zuverlässigkeit und Performance zu verbessern. 
	\par\smallskip 
	Einer dieser Arbeitsschritte ist die Segmentierung, welche unterschiedliche 
	Bereiche in einem Eingabebild ausfindig macht und für weitere 
	Systemkomponenten klar erkennbar voneinander abgrenzt. Im Rahmen dieser 
	vorliegenden Arbeit soll eine Methode aus den Reihen der Evolutionären 
	Algorithmen im Bezug auf die Segmentierung erprobt werden. Als 
	Versuchsobjekte dienen digitale Bilder von metallischen Platten, worauf ein 
	Zeichencode graviert ist, der erkannt werden soll.