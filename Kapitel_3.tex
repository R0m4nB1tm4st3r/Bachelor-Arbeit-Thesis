\glsreset{de}
\chapter{Der Lösungsansatz aus den \gls{ea} - \gls{de}}
\label{sec:ex}

	\section{\gls{ea}}
	\label{sec:evol}
	
	In der September-Ausgabe des Magazins \textit{Spektrum der Wissenschaft} \cite{j-h-holland} aus dem Jahr 1992 ließen sich folgende Worte von John H. Holland lesen: \\
	
	\begin{quote}
		\textit{Lebewesen sind vollendete Problemlöser. In der Vielzahl der Aufgaben, die sie bewältigen, übertreffen sie die besten Computerprogramme bei weitem - zur besonderen Frustration der Programmierer, die Monate oder gar Jahre harter geistiger Arbeit für einen Algorithmus aufwenden, während Organismen ihre Fähigkeiten durch den scheinbar ziellosen Mechanismus der Evolution erwerben. \\}
	\end{quote}
	
	Dieses Zitat, welches die Einleitung von \cite{ger-kla-kru-intro} bildet, bietet bereits eine grobe Vorstellung vom Wesen und der Herkunft von \gls{ea}. In den meisten literarischen Werken zu diesem Thema - beispielsweise \cite{ger-kla-kru-intro, eib-smi-ea} - wird festgehalten, dass verschiedene Kategorien von \gls{ea} existieren. Allerdings ist diese Einteilung laut \cite{eib-smi-ea} lediglich historischer Natur und verschwimmt zunehmend. Ursächlich hierfür ist, dass der einzige Unterschied zwischen den einzelnen Varianten in der Repräsentation\footnote{mögliche Darstellungsformen sind - wie in \cite{eib-smi-ea} beschrieben - Strings (\textbf{Genetische Algorithmen}), Vektoren von (reellen) Zahlen (\textbf{Evolutionsstrategien}) und Baumstrukturen (\textbf{Genetische Programmierung}) sowie endliche Automaten (\textbf{Evolutionäre Programmierung})} der Lösungskandidaten (hierzu in Kürze mehr) liegt. Die Funktionsweise, welche allen \gls{ea} zugrunde liegt, ist dieselbe: \\
	
	\begin{itemize}
		\item Gegeben sei eine Liste von Lösungskandidaten (im Folgenden \textbf{Population} genannt) als Input für eine Zielfunktion (fortlaufend als \textbf{Fitnessfunktion} bezeichnet), die in der Regel minimiert oder maximiert werden soll.
		\item Des weiteren sei diese Population von der Generation $n$.
		\item Zunächst wird die Fitness jener n-ten Population mithilfe der Fitnessfunktion ausgewertet und zwischengespeichert. 
		\item Aus dieser wird mittels verschiedener Variationsoperationen (\textbf{Rekombination}, \textbf{Mutation}) eine neue Population  erzeugt.
		\item Im nächsten Schritt wird diese ebenfalls durch die Fitnessfunktion evaluiert.
		\item Um anschließend eine Population der Generation $n+1$ zu generieren, deren Fitness größer gleich der $n$-ten Population ist, wird eine \textbf{Selektion}soperation durchgeführt, die aus den beiden vorliegenden Populationen jeweils die Kandidaten mit der besten Fitness auswählt.
	\end{itemize}
	
	Die soeben beschriebene Vorgehensweise kann so lange wiederholt werden, 
	bis entweder eine gewünschte Lösung vorliegt oder ein (vorher 
	festgelegtes) Iterationslimit erreicht ist. Im folgenden Abschnitt 
	werden all diese Aspekte anhand des \gls{de} Algorithmus, welcher 
	ebenfalls im Rahmen dieser Arbeit verwendet wird, deutlich.\\
	
	Vorrangig werden \gls{ea}

	\glsreset{de}

	\section{\gls{de}}
	\label{sec:de}

		Wie in Abschnitt \ref{sec:evol} bereits angedeutet, reiht sich \gls{de} in die \gls{ea} ein. Der von Rainer Storn und Kenneth Price \cite{storn-price-de} 