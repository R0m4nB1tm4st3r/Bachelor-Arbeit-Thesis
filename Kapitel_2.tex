\chapter{Die Problemstellung}
\label{sec:prob}
	
	Dieses Kapitel soll eine grobe Einführung in die Thematiken Bildverarbeitung und \gls{ocr} bieten und ferner die Motivation für diese Arbeit im Zusammenhang mit der Segmentierung von Bildern aufzeigen.
	\section{Grundlegendes über Bildverarbeitung}
	\label{sec:bild-basics}
	
		Zunächst steht die Frage nach einer (allgemein gültigen) Definition des Begriffes Bildverarbeitung im Raum. Hierzu gehen die Meinungen unter den Fachleuten auseinander. Präzise ausgedrückt, liegt die Meinungsverschiedenheit in der Frage nach der Grenze zwischen der Bildverarbeitung und damit verwandten Bereichen, z.B. dem maschinellen Sehen \cite[S. 2]{gonzalez-woods}. Zum Zwecke der Veranschaulichung soll die in \cite[S. 2]{gonzalez-woods} gegebene Definition ausreichen. Danach soll zu Beginn eine Unterscheidung von rechnergestützten Prozessen wie folgt vorgenommen werden:
		\begin{description}
			\item[Low-Level] Darunter fallen einfache Operationen direkt am Bild, wie zum Beispiel Rauschreduktion oder die Erhöhung des Kontrastes, oder aber auch simple arithmetische Operationen. Input sowie Output sind jeweils durch ein Bild gegeben.
			\item[Mid-Level] Diese Kategorie von Prozessen zeichnet sich dadurch aus, dass aus einem Bild bestimmte Eigenschaften/Bereiche extrahiert werden, um daraus eine Information zu gewinnen. Während die Eingabe aus einem ganzen Bild besteht, beläuft sich die Ausgabe lediglich auf einen Teil davon bzw. ein bestimmtes Attribut des Bildes. Exemplarisch hierfür kann die Segmentierung angeführt werden, die in dieser Arbeit mit Hilfe der in Abschnitt \ref{sec:de} vorgestellten Methode realisiert werden soll. Bei der Segmentierung wird das Bild in Sub-Bilder eingeteilt, bis die für die jeweilige Anwendung interessante Region klar abgegrenzt im Bild erscheint.
			\item[High-Level] Hier werden die aus den Mid-Level Prozessen erhaltenen Informationen verwendet, um bestimmte Aktionen auszulösen (z.B. im Bezug auf maschinelles Sehen).
		\end{description}
		Auf dieser Grundlage lassen sich nun die beiden erstgenannten Sorten von Prozessen zusammenfassen unter dem Terminus Bildverarbeitung. \\
		In Anlehnung hieran kann ein Bild aus mathematischer Sicht als eine Funktion $f(x, y)$ von zwei Variablen begriffen werden, die an der Stelle $(x, y)$ ein Ergebnis zurückliefert, das als \textit{Intensität} interpretiert wird. Diese kann, je nach Farbsystem, entweder ein Skalar (schwarz-weiß) oder ein Tupel (\textbf{R}ed-\textbf{G}reen-\textbf{B}lue: 3-Tupel \cite{rgb-info}) sein \cite[S. 50]{gonzalez-woods}. \\
		Um der Vollständigkeit genüge zu tun, wird in Abbildung \ref{fig:indust-imgproc} der hardwaretechnische Aufbau eines Bildverarbeitungssystems dargestellt:
		\begin{figure}[H]
			\includegraphics[width=\linewidth]{Industrielles-Bildverarbeitungssystem}
			\caption{typischer Aufbau eines Bildverarbeitungssystems (nachgezeichnet aus \cite[S. 7]{indust-imgproc})}
			\label{fig:indust-imgproc}
		\end{figure}
		\color{red} [Bild beschreiben] \color{black}
		\vfill
		\glsreset{ocr}
	\section{\gls{ocr} und Segmentierung}
		\begin{figure}[H]
			\centering
			\includegraphics[width=\linewidth]{Ablauf-OCR_Cheriet-et-al.pdf}
			\caption[typisches \gls{ocr}-Ablaufschema]{Schematische Darstellung des 
				Ablaufs in einem intelligenten \gls{ocr}-System (Nachbildung aus 
				\cite[Seite 7]{cher-et-al-ocr})}
			\label{fig:ocr-system}
		\end{figure}
		
		Einen in der heutigen Zeit zentralen Anwendungsfall für die (industrielle) Bildverarbeitung stellt \gls{ocr} dar. Wie der Name bereits vermuten lässt, ist das Ziel der \gls{ocr}, Zeichen oder Formen aus digitalen Bildern zu extrahieren und weiterzuverarbeiten. Schon in den frühen 1950ern haben Forscher nach Möglichkeiten gesucht, auf Papier befindlichen Text mit Hilfe von Rechnerkapazitäten einzulesen, um dem Menschen die mühselige Arbeit des manuellen Abtippens von Dokumenten abzunehmen \cite{cher-et-al-ocr}.Dieser Vorgang lässt sich in vier Phasen unterteilen (blau gefärbte Rechtecke in Abbildung \ref{fig:ocr-system}): Während der \textbf{Layout Analysis} wird das noch rohe Bild so bearbeitet, dass lediglich der relevante Bereich mit der zu lesenden Information zu sehen ist. Anschließend wird das Bild beim \textbf{Pre-Processing} gegebenenfalls von Störeffekten wie Rauschen oder Artefakten befreit, welche die Erfolgschance in der nächsten Phase mindern. Darauf aufbauend geschieht in der nun folgenden \textbf{Character Recognition} die eigentliche Zeichenerkennung. Hier findet sich auch die Segmentierung wieder, die für eine erfolgreiche Zeichenerkennung von elementarer Bedeutung sowie ein zentraler Bestandteil dieser Arbeit ist. Schließlich erfolgt im Zuge des \textbf{Post-Processing} eine Validierung des zuvor gelesenen Textes und eventuell eine weitere Verarbeitung der gewonnenen Information. \\
		
		Wie im obigen Absatz bereits festgehalten, kann eine erfolgreiche Zeichenerkennung ohne eine gute Segmentierung nicht funktionieren. In Abbildung \ref{fig:seg-example-good} lässt sich an einem Beispiel betrachten, was die Segmentierung eines Bildes bewirkt, wobei dies ein Beispiel für eine erfolgreiche Segmentierung ist:
		\begin{figure}[H]
			\centering
			\subfloat[][unsegmentiertes OriginalBild]{\includegraphics[width=0.48\linewidth]{IMG_3909245}}
			\qquad
			\subfloat[][segmentiertes Bild]{\includegraphics[width=0.48\linewidth]{IMG_3909245-seg-good}}
			\caption{Segmentierung an einem Beispiel (Einteilung in zwei Klassen) - gutes Beispiel}
			\label{fig:seg-example-good}
		\end{figure}
		Dem gegenübergestellt wird ein Exempel einer erfolglosen Segmentierung in Abbildung \ref{fig:seg-example-bad}:
		\begin{figure}[H]
			\centering
			\subfloat[][unsegmentiertes OriginalBild]{\includegraphics[width=0.48\linewidth]{IMG_3909236}}
			\qquad
			\subfloat[][segmentiertes Bild]{\includegraphics[width=0.48\linewidth]{IMG_3909236-seg-bad}}
			\caption{Segmentierung an einem Beispiel (Einteilung in zwei Klassen) - schlechtes Beispiel}
			\label{fig:seg-example-bad}
		\end{figure}
		Wie auf Abbildung \ref{fig:seg-example-bad}b zu erkennen ist, werden zu viele Pixel in der Umgebung des ersten und letzten Zeichens schwarz gefärbt, was die Konturen dieser beinahe verschwinden lässt. Zwar kann sich mit dem menschlichen Auge der Text noch erkannt werden, aber für eine Maschine wird es aus diesem Bild unmöglich, den darauf befindlichen Text zu lesen.\\
		Aus der bereits langjährigen Geschichte der \gls{ocr} können inzwischen zahlreiche Arten gefunden werden, um eine Segmentierung zu realisieren. Diese sind in drei Gruppen unterteilbar \cite[Kapitel 10]{gonzalez-woods}: \textit{Thresholding}, \textit{Edge-Based-Segmentation} und \textit{Region-Based-Segmentation}. Sowohl aus erstgenannter als auch aus letztgenannter Gruppe von Segmentierungsmethoden wird jeweils ein Modellansatz in den Paragraphen \ref{sec:meth1} und \ref{sec:meth2} eingehend vorgestellt sowie im weiteren Verlauf dieser Arbeit zur Generierung der Ergebnisse in Kapitel \ref{sec:results} herangezogen. \\
		
		Um der Vollständigkeit genüge zu tun, sei an dieser Stelle die Segmentierungsmethode nach Otsu erwähnt, die in der heutigen industriellen Bildverarbeitung noch Anwendung findet. Hierbei wird das Thresholding-Problem aus der statistischen Sicht betrachtet, mit dem Ziel, den durchschnittlichen Fehler durch das Zuteilen eines Pixels zu zwei oder mehreren Klassen zu minimieren. \color{red} [Fortsetzung folgt] \color{black}

		\subsection{Modell 1: Gaußsche Annäherung an das Histogramm}
		\label{sec:meth1}
			
			Dieses Vorgehen aus der Gruppe der \textit{Thresholding}-Methoden basiert auf der Arbeit von Erik Cuevas und seinen Kollegen \cite{cuevas-meth1} und zielt darauf ab, das Histogramm eines Bildes mittels einer Summe von Gaußschen Wahrscheinlichkeitsfunktionen zu approximieren und den Fehler zwischen diesen beiden Funktionen zu minimieren.
			
			Unter der Annahme, dass ein Graustufenbild vorliegt - also die in Abschnitt \ref{sec:bild-basics} erwähnte Bildfunktion $f(x,y)$ einen Wert im Intervall $[0, ..., L-1]$ zurückliefert, mit $L = $ Anzahl der Graustufen - ergibt sich das Histogramm eines Bildes nach Gleichung \ref{eq:histogram} zu
			\begin{flalign}
				\centering
				h(g) &= \frac{n_{g}}{N} \label{eq:histogram}
			\end{flalign}
			dabei steht $n_{g}$ für die Anzahl an Pixel mit der Graustufe $g$, während $N$ die Gesamtzahl aller Pixel im Bild repräsentiert.\\
			Dieses Histogramm kann dann näherungsweise über die Summe von Gaußschen Wahrscheinlichkeitsdichtefunktionen, wie in Gleichung \ref{eq:gauss-sum} dargestellt, angegeben werden:
			\begin{flalign}
				\centering
				p(x) = \sum_{i=1}^{K} P_{i} \cdot p_{i}(x) \quad \textrm{mit} \quad p_{i}(x) = \frac{1}{\sigma_{i}\sqrt{2\pi}} \exp\left(-\frac{1}{2} \cdot \left(\frac{x - \mu_{i}}{\sigma_{i}}\right)^{2}\right), \label{eq:gauss-sum}
			\end{flalign}
			unter der Voraussetzung, dass $K$ die Anzahl der Klassen, in die das Bild aufgeteilt werden soll, und $P_{i}$ die Wahrscheinlichkeit dafür darstellt, dass ein zufällig gewählter Grauwert $x$ innerhalb der Klasse $i$ liegt. Weiterhin ist $p_{i}(x)$ die der Klasse $i$ zugehörige Komponente der Approximation $p(x)$ in Form einer Gaußschen Dichtefunktion, jedoch wurde die Formel aus \cite{papula-gauss} entnommen, da die Darstellung in \cite{cuevas-meth1} fehlerhaft ist (das $\sigma_{i}$ im ersten Bruch befindet sich innerhalb der Quadratwurzel). Zusätzlich ist $\sigma_{i}$ als Standardabweichung und $\mu_{i}$ als Mittelwert angegeben.\\
			Nun sind die Parameter $\sigma_{i}$, $\mu_{i}$ und $P_{i}$ mit $i = 1, ..., K$ so zu bestimmen, dass nachstehendes Problem
			\begin{flalign}
				\centering
				\underset{P_{i}, \mu_{i}, \sigma_{i}}{arg \ min} \ E \quad \textrm{mit } E = \frac{1}{n} \sum_{j=1}^{n}[p(x_{j}) - h(x_{j})]^{2} + \omega \cdot \left| \left( \sum_{i=1}^{K} P_{i} \right) - 1 \right| \label{eq:mean-square-error}
			\end{flalign}
			gelöst wird - ausgehend von der Graustufenanzahl $n$ und dem Penalty-Faktor $\omega$, der sich genau dann auswirkt, wenn die Bedingung $\sum_{i=1}^{K} P_{i} = 1$ nicht erfüllt ist.\\
			Die Minimierung einer solchen Funktion erweist sich laut \cite{cuevas-meth1} als ein komplexes Problem. Eine analytische Lösung erfordert, wie in \cite[S. 248-250]{papula-optimization} anschaulich dargelegt, die jeweiligen partiellen Differentiationen der Funktion in Gleichung \ref{eq:mean-square-error} nach den Parametern $\sigma_{i}$, $\mu_{i}$ und $P_{i}$ zu bestimmen und gleich Null zu setzen. Dies führt auf ein Gleichungssystem mit $3K$ Unbekannten und $3K$ Gleichungen der Form
			\begin{flalign*}
				E_{\sigma_{l}} &= \frac{\partial E}{\partial \sigma_{l}}, & E_{\mu_{l}} &= \frac{\partial E}{\partial \mu_{l}}, & E_{P_{l}} &= \frac{\partial E}{\partial P_{l}}, & l &= 1, ..., K
			\end{flalign*}
			Nach den in \cite[S. 26]{papula-alg-eq} aufgestellten Kriterien für die Linearität eines Gleichungssystems ist obiges ein \textit{nichtlineares} Gleichungssystem und somit analytisch bestenfalls mit erheblichem Aufwand lösbar \color{red}[muss noch belegt werden]\color{black}. In solchen Fällen wird dann ein numerisches Verfahren zur Approximation einer Lösung in annehmbarer Zeit herangezogen, beispielsweise jenes, das in Abschnitt \ref{sec:de} behandelt wird. 
			
			Im Folgenden soll zur weiteren Erläuterung des Modells davon ausgegangen werden, dass die Kenngrößen $\sigma_{i}$, $\mu_{i}$ und $P_{i}$ im Sinne von Gleichung \ref{eq:mean-square-error} bereits ermittelt wurden. \\
			Der letzte Schritt, um hieraus schließlich die Threshold-Werte $T_{1}, ... , T_{K-1}$ zu erhalten, führt über die Lösung der folgenden quadratischen Gleichung:
			\begin{flalign}
				AT_{i^{2}} + BT_{i} + C &= 0 \label{eq:quad-eq-thresh}
			\end{flalign}
			mit den Koeffizienten
			\begin{flalign}
				A &= \sigma_{i}^{2} - \sigma_{i+1}^{2}, \notag \\
				B &= 2 \cdot (\mu_{i}\sigma_{i+1}^{2} - \mu_{i+1}\sigma_{i}^{2}), \notag \\
				C &= (\sigma_{i}\mu_{i+1})^{2} - (\sigma_{i+1}\mu_{i})^{2} + 2 \cdot (\sigma_{i}\sigma_{i+1})^{2} \cdot \ln \left(\frac{\sigma_{i+1}P_{i}}{\sigma_{i}P_{i+1}}\right) \notag
			\end{flalign}
			Wie sich diese im Einzelnen zusammensetzen, soll an dieser Stelle nicht erläutert werden, lässt sich jedoch bei \cite[S. 13+14]{cuevas-meth1} nachlesen.
			
			Anhand der so ermittelten Threshold-Werte werden nun alle Pixel des Bildes mittels untenstehender Gleichung aus \cite[S. 739]{gonzalez-woods} zugeordnet:
			\begin{flalign}
				\centering
				g(x, y) = 
				\begin{cases}
					a_{1} & \textrm{if } f(x, y) > T_{K-1}\\
					a_{2} & \textrm{if } T_{K-2} < f(x, y) \leq T_{K-1}\\
					... \\
					a_{K} & \textrm{if } f(x, y) \leq T_{1}
				\end{cases}
				\label{eq:threshold}
			\end{flalign}
			Dabei sind $a_{1}, ..., a_{K}$ voneinander unterschiedliche Graustufenwerte.
	
		\subsection{Modell 2: K-Means}
		\label{sec:meth2}
		
			Der K-Means-Algorithmus wurde 1967 von James B. MacQueen \cite{macqueen-kmeans} vorgestellt und wird heute noch häufig zur Gruppierung großer Datenmengen eingesetzt - so in \cite{kmeans-info} zu lesen.\\
			Die grundlegende Idee hinter K-Means besteht in der Generierung von Mittelpunkten als Repräsentanten für verschiedene Bereiche in einem Datensatz. Formal ausgedrückt bedeutet dies \cite{mozdren-meth2}: \\
			Gegeben sei eine Folge von Messpunkten $x_{j} \in \mathbb{R}^{d},$ $j \in [1,n]$, wobei $n$ gleich der Anzahl der Messpunkte ist und $d$ für die Dimension dieser steht. Dann existiert eine Segmentierung $S = \{S_{1}, ..., S_{K}\}$ mit den für die einzelnen Segmenten $S_{i}$ stellvertretenden Mittelpunkten $\{\mu_{i}, ..., \mu_{K}\}$, die der Lösung des anschließenden Problems \ref{eq:kmeans-min} genügt:
			\begin{flalign}
				\centering
				\underset{S}{arg \ min} \sum_{i=1}^{K} \sum_{x_{j} \in S_{i}}\left|\left|x_{j} - \mu_{i}\right|\right|^{2}, \quad \left|\left|x_{j} - \mu_{i}\right|\right|^{2} = \textrm{euklid. Distanz zwischen } x_{j} \textrm{ und } \mu_{i}
				\label{eq:kmeans-min}
			\end{flalign}
			Dabei erfolgt die Eruierung der Mittelpunkte in zwei Schritten:\\
			Im \textit{Assignment-Step} (Gleichung \ref{eq:kmeans-assign}) werden die Datenpunkte den jeweiligen Segmenten. Daraufhin ergeben sich im \textit{Update-Step} (Gleichung \ref{eq:kmeans-update}) die neuen Mittelpunkte aus den arithmetischen Mittelwerten innerhalb der Segmente:
			\begin{flalign}
				S_{i}^{(t)} &= \left\{ x_{a} : \left|\left|x_{a} - \mu_{i}^{(t)}\right|\right|^{2} \leq \left|\left|x_{a} - \mu_{j^{(t)}}\right|\right|^{2} \forall j \in \{1, ..., K\} \textbackslash \{i\}  \right\} \label{eq:kmeans-assign}\\
				\mu_{i}^{(t+1)} &= \frac{1}{\left| S_{i}^{(t)} \right|} \sum_{x_{j} \in S_{i}^{(t)}} x_{j}, \label{eq:kmeans-update}
			\end{flalign}
			mit $t$ als aktueller Iterationsindex. Diese Abläufe werden wiederholt, bis die Mittelpunkte $\mu_{i}$ sich nicht weiter wesentlich ändern.